\documentclass[a4paper,11pt,titlepage]{article}
\usepackage{tabularx,booktabs,url}
\usepackage[dvips]{graphicx}
\usepackage{amsmath}
%\usepackage[nolists]{endfloat}
\usepackage{vmargin,setspace}
\usepackage[longnamesfirst,sort]{natbib}
\usepackage{fancyhdr}
\usepackage{comment,subfigure}
\usepackage[nodayofweek]{datetime}
\usepackage[draft,ulem=normalem]{changes}
\usepackage{url}
\usepackage{color}
\specialcomment{db}{\begin{textcolor}{red}}{\end{textcolor}}

\definechangesauthor[color=red]{db}

\usepackage[sc]{mathpazo}
\linespread{1.05}         % Palatino needs more leading (space between lines)
\usepackage[T1]{fontenc}

\widowpenalty10000

\newcolumntype{Y}{>{\small\raggedright\arraybackslash}X}
\newcolumntype{C}{>{\centering}X}
\pagestyle{fancy}
\lhead{}
\chead{}
\rhead{}
\cfoot{\thepage}
\lfoot{}
\rfoot{}
\renewcommand{\headrulewidth}{0pt}

\singlespace
 \setpapersize{A4}
 \setmarginsrb{25mm}{25mm}{25mm}{25mm}{12pt}{10mm}{0pt}{10mm}
\let \citeasnoun\citet
\let\cite\citep
\let\citename\citeauthor
\newcommand{\possessivecite}[1]{%
   \citeauthor{#1}'s\ \citeyearpar{#1}}
\bibpunct{(}{)}{;}{a}{}{,}
\newcommand{\s}{\rlap{$^*$}}
\renewcommand{\ss}{\rlap{$^{**}$}}

\title{Comparing public and private sector pay across socio-economic groups}
\author{David N. Barron\\ Sa\"{\i}d Business School\\ University of Oxford}
\date{\today}

\bibliographystyle{ajs}

\begin{document}

\maketitle

\thispagestyle{fancy}

\begin{abstract}
    There has been a lot of recent interest in the relationship between public and private sector pay.  The difficulty with such comparisons is that the occupational composition of the public and private sectors is quite different.  In this paper I explore the difference between public and private pay within various socio-economic groups. I find that the often-reported public sector premium virtually disappears for men using this approach.  However, there is still a significant public sector premium for women in most socio-economic groups.
\end{abstract}

\section{Introduction}
Comparing the levels of pay earned by employees in the public and private sectors is of considerable interest in the media, government and trade unions.  For example, the latter often claim that comparatively generous public sector pensions can be justified in part because of the relatively low basic pay of employees in this sector.  However, research into the issue generally finds that the wages of a ``typical'' public sector employee are higher than those of a ``typical'' private sector worker.  A recent report, for example, put the public sector pay premium at over 40 per cent \citep{Holmes2011}.  Similarly, the recent decision by the UK government to move away from national pay bargaining in the public sector has been justified with reference to the relatively large public sector pay premium that is observed in some parts of the country \cite{OME2012} . This phenomenon is held to distort local labour markets, in that private sector organizations will not be able to attract the best employees and public sector employment will be lower than it would be if public sector pay were as responsive to local market conditions as is private sector pay.

\begin{figure}[htb]
    \centering
    \includegraphics[scale=.5]{Occupations.pdf}
    \caption{Proportion of employees in socio-economic groups\label{fig:segs}}
\end{figure}

 One problem with this type of research is that the public and private sector workforces are very different \citep{IDS2011}.  Figure \ref{fig:segs} shows a comparison of the proportion of employees in the public and private sector in selected occupations using data from the British Household Panel Survey (BHPS).  Comparing the average earnings of a public sector employee with the average earnings of a private sector employee will therefore not be very informative, as the proportion of the variance in an individual's earnings explained by his or her socio-economic group is much larger than the proportion explained by whether he or she works in the public or private sectors.

 Other authors using other sources of data have reported similar differences.  For example, \citet{Damant2011}
 find that the public sector is made up of a higher proportion of highly skilled jobs, a difference that has grown in recent years as more public authorities have outsourced low skilled jobs to the private sector.  They also find that the public sector workforce is, on average, older than that of the private sector.  Given that earnings tend to increase with age (and hence experience), this difference is also likely to explain some of the difference between public and private sector wages.  Indeed, \citet{Damant2011} find that, if one restricts one's attention only to employees who are graduates, public sector workers are paid 5.7 per cent \emph{less} than graduates in the private sector.  However, this report estimates the public sector premium in 2010 to be 7.8 per cent after controlling for age, qualifications, region of the country and occupation (as defined by the Standard Occupational Classification 2000 \citep{SOC2000}).

 The purpose of this paper is to estimate a separate public sector pay premium for men and women in each of nine socio-economic groups using data obtained from the British Household Panel Survey (BHPS).  This approach enables us to investigate in much more detail the differences between public and private sector pay, controlling for age, education, experience and, because we are using panel data, for unobserved heterogeneity. I used the National Statistics Socio-economic Classification to define socio-economic groups \citep{ONSSEC}.  This classification is a measure of ``the employment relations and conditions of occupations''. More precisely, is is based on labour market situation, which ``equates to \dots economic security and prospects of economic advancement''; and  work situation, which ``refers primarily to location in systems of authority and control at work, although degree of autonomy at work is a secondary aspect'' \citep{ONSSEC}. In this analysis, only people in employment are included: self-employed and own-account workers are excluded.

 The reason for using this approach is that the occupations in the same socio-economic group can be thought of as equivalent in these important respects. This is essentially a compromise between comparing specific occupations---very few of which have large numbers of people in both the public and private sectors---and pooling all occupations together and controlling only for individual employee differences.  Separate analysis of people in different socio-economic groups means that we are comparing people in occupations that can be thought of as very similar to each other with the exception that some are in the private sector and some in the public sector. This should enable additional insights into the nature of any systematic differences in the wages earned by people in these two sectors.

\section{Literature}
A number of recent publications have investigated the size of the difference in average earnings between people employed in the public and private sectors.  A report by the Policy Exchange \citep{Holmes2011} created a lot of attention in the media with its central claim that there was a 24 per cent pay premium for working in the public sector in the UK in 2010, with this rising to 43 per cent if the value of pensions was taken into account.  Less widely reported at the time was that this raw difference declined to 8.8 per cent when factors like age, experience and qualifications were taken into account.  However, as some commentators on this report noted, a problem with such comparisons is finding meaningful comparators:
\begin{quote}
    Although individuals in the public and private sectors may sometimes be able to identify similar roles within their respective organisations, the sectors as a whole have very different profiles. The public sector employs a higher proportion of professionally-trained staff, undertaking specific service roles such as those within healthcare, education and the emergency services. As a result, a much larger percentage of the public sector workforce is educated to diploma or degree level \citep[p.~13]{IDS2011}
\end{quote}
These differences certainly account for a large proportion of the observed raw differences in earnings between the two sectors.  Consequently, a number of studies have attempted to control for variation in employees, either by explicitly modelling factors associated with human capital (for example, education and experience) or, when panel data are used, controlling for individual heterogeneity by means of fixed effects.

\citet{Yu2005} use quantile regression on panel data and include controls for years of education and work experience to obtain their estimates of a public sector pay premium for male employees in full time employment.  In 2001 they found that public sector workers in the lowest decile of the wage distribution enjoyed a pay premium of 13.2 per cent.  However, this premium declined as wages increased, with public sector workers at the median experiencing a pay \emph{penalty}, with the penalty increasing to 11.7 per cent at the top decile of the wage distribution.  Overall, these differences virtually ``cancelled out,'' so a standard regression model would have estimated the public sector pay premium at 0.25 per cent.

\citet{Luciflora2006} review a number of previous studies and find that the average estimated public sector wage premium is ``close to 5 percent, although  it  is  much  higher  for  females  (15–-18  percent)  as  compared  to
men (2-–5 percent)'' (p.~48).  However, about half of this difference can be explained by observed individual differences, such as education.   Using data from a single year (1998) from the Labour Force Survey and quantile regression, \citet{Luciflora2006} find that the public sector pay premium declines monotonically as pay increases, although it does not become a penalty until one reaches the top decile.  They also find that the public sector premium is much greater for women than for men at all wage levels.

\citet{Disney2008} estimate the public sector wage premium (or penalty) using a fixed effects approach, but also adopt a method that allows the size of the premium to change over time.  This is particularly important given that they study the public sector premium over more than thirty years (1975--2006), and so an assumption that the premium was constant would be difficult to justify.  They obtain separate estimates for the premium for men and women, but otherwise assume that it is the same for all employees regardless of occupation.

\citet{Chatterji2007} restrict their attention to public sector premia among male employees, and use a cross-sectional research design based on the Workplace Employee Relations Survey 2004.  They find a raw public sector wage premium of 11.7 per cent.  However, they also find that only 1.6 percent of this premium cannot be explained by individual and occupational characteristics, of which the level of formal education was the most important. Among highly skilled employees, they find evidence that public sector workers experience a pay \emph{penalty} of 5.5 per cent when individual and occupational characteristics are taken into account.  On the other hand, unskilled employees in the public sector enjoy a pay premium of 7.2 per cent.

The evidence from previous research, then, shows a number of consistent results. First, women enjoy a public sector pay premium that is considerably larger than that experienced by men.  Second, the pay premium is larger at the lower ends of the pay distribution, and may become a pay penalty at the higher end of the distribution.  Finally, the size of the pay premium (or penalty) is not constant, but has varied significantly over time.  These results have been obtained using a variety of data sources and methods of estimation.  To date, however, researchers have not used panel data to investigate variations in the size of the public sector pay gap among different socio-economic groups.

\section{Theory}

Why might wages in the public and private sectors differ? There are a number of possible explanations.  First, in the UK a large proportion of public sector employees' pay is set at a national level.  This might be expected to give employers monopsonistic power and hence depress wages, as has been suggested by a number of authors \citep{Boal1997}.  Evidence on this point is mixed, however, and clearly the monopsony is not perfect given that there are typically private sector counterparts to public sector employees like nurses, teachers and doctors who on average earn more than their public sector colleagues \citep{Barron2012}. Additionally any market structure effect may be offset by the fact that levels of trade union membership are greater in the public than the private sector and bilateral pay bargaining is generally carried out a national level.  Given this institutional framework, it is not clear where the balance of power will lie, and hence it is not clear whether one would predict a public sector premium or penalty.

Also possibly relevant is the idea that employees in some occupations found predominantly in the public sector in the UK, such as nursing and teaching, benefit employees in non-pecuniary ways in addition to the pecuniary benefits they receive.  Such employees are intrinsically motivated to deliver high quality service, and their greater ability to do so in the public sector would mean that they would still prefer to work in the public sector even if they could earn higher wages in the private sector.  This would lead one to predict higher wages in the private sector for a given occupation.

The often reported higher levels of public sector pay premium for women than men raises the question of whether there are differences in the levels of discrimination faced by women in the two sectors \citep{Byron2010}.  Again, it is not clear what theory would predict.  On the one hand, if there is a greater commitment to the use of formalized procedures for recruitment, job evaluation, and so on in the public sector, we might expect to see lower levels of discrimination here \citep{Kaufman2002}. Bureaucratic formalization has been referred to by some scholars as ``the great leveller'', but feminist critiques have suggested that bureaucratic structures can be used to legitimize gender differences or even that they are inherently patriarchal and hence create gender inequalities \citep{Baron2007}.   On the other hand, competitive pressures in the private sector might mean that inefficiencies in recruitment, retention and deployment of staff due to discrimination are less likely to persist \citep{Becker1971}.  Empirical evidence is, however, mixed, with some studies suggesting there is indeed less pay discrimination in the private sector, but others finding evidence of greater levels of discrimination \citep{Byron2010}.

\section{Data and methods}

The data for our study come from the British Household Panel Survey (BHPS)  \citep{Taylor2010}.  The BHPS began in 1991 and has been carried out annually every since; we use data from the first eighteen waves.  The original survey included a nationally representative sample of over 5000 households.  All members of these households over the age of sixteen were included, making a total of about 10,000 individuals.  These people have been included in all subsequent waves, as have any new adult members of the original households and new households formed by members of the original panel.

\begin{table}[htb]
\caption{Socio-economic group classification.  \label{tab:SEG}}
\begin{center}
\begin{tabular}{lr}
\toprule
  Higher management & 4818 \\
  Higher professional & 7984 \\
  Lower professional & 19099 \\
  Lower managerial & 10686 \\
  Higher supervisory & 3910 \\
  Intermediate occupations & 21386 \\
  Lower technical & 14432 \\
  Semi-routine & 24642 \\
  Routine & 18655 \\
\bottomrule
\end{tabular}
\end{center}
\end{table}

We carry out our regressions on different socio-economic groups (SEG) separately. The socio-economic group variable that we use is the National Statistics Socio-Economic Classification (NS-SEC). The descriptions of the categories and the distribution of people in them are shown in table \ref{tab:SEG}.  Classification of BHPS respondents into NS-SEC categories is performed using the Standard Occupational Classification (SOC) of their job, as described in \citet{Taylor2010}.

The model of wages that we estimate is a random intercept model, as shown in equation \eqref{eq:wage}:

\begin{gather}
y_{it} = x_{it} \beta + \alpha_i + u_{it}; \notag\\
\alpha_i \sim N(0,\sigma_\alpha^2); \label{eq:wage}\\
u_{it} \sim N(0,\sigma_u^2),\notag
\end{gather}
where $y_{it}$ is the usual log hourly wage of employee $i$ in year $t$, $x_{it}$ are explanatory variables, $\beta$ is a vector of regression parameters to be estimated, and $\alpha$ and $u$ are error terms with the properties shown.

Explanatory variables include a measure of the highest level of education obtained by the employee (secondary school qualification or tertiary education qualification, with no qualification being the excluded category), the number of years the employee has been in his or her current job, age in years and its square, dummy variables for the wave of the survey, sex, an indicator of whether the employee works in the public sector, and an interaction between sex and public sector.  When testing for stability of the public sector pay gap over time we also included interaction between the wave dummies and the public sector indicator.  Estimates were obtained using the \texttt{lmer} function from the \texttt{lme4} package in R \citep{R2011,lme2011}

\section{Descriptive statistics}
Over the full 18 waves included in the analysis, the median hourly wage of public sector employees is \pounds 9.68, which is 29 per cent higher than the \pounds 7.51 that is the median hourly wage of employees in the private sector.  The wage distributions on a log scale are shown in Figure \ref{fig:density}.  The standard deviation of log hourly wage for private sector employees is $.60$, while it is $.53$ for the public sector.  The lower dispersion of wages in the public sector is a consistent finding of previous research, generally explained by reference to the high levels of remuneration earned by the top earners in private sector occupations such as banking.  The mean hourly earnings of the nine socio-economic groups are shown in table \ref{tab:wages}.

\begin{table}
\caption{Mean hourly earnings of each socio-economic group \label{tab:wages}}
\begin{center}
\begin{tabular}{lr}
\toprule
Socio-economic group & Mean hourly wage (\pounds)\\
\midrule
  Higher management & 10.45 \\
  Higher professional & 10.49 \\
  Lower professional & 9.98 \\
  Lower managerial & 10.22 \\
  Higher supervisory & 10.04 \\
  Intermediate occupations & 9.81 \\
  Lower technical & 9.52 \\
  Semi-routine & 9.59 \\
  Routine & 9.15 \\
\bottomrule
\end{tabular}
\end{center}
\end{table}

\begin{figure}[tb]
    \centering
    \includegraphics[scale=.5]{Density.pdf}
    \caption{Kernel density estimates of the distribution of public and private sector wages\label{fig:density}}
\end{figure}

A better estimate of the public sector pay premium can be obtained by using multilevel regression.  Using this method gives an estimate of the public sector pay premium of $20.3$ per cent.  If we include dummy variables for year and also interact these dummies with the sector dummy variable, we obtain estimates of the change in the premium over time.  These results are summarised in Figure \ref{fig:premium}, which shows that the public sector pay premium declined during the 1990s, but has been reasonably constant since the turn of the century.

\begin{figure}[tb]
    \centering
    \includegraphics[scale=.5]{Premium.pdf}
    \caption{How the public sector pay premium has changed over time\label{fig:premium}}
\end{figure}

Previous studies have shown that there is a bigger difference between the public and private sector earnings of women than men.  Figure \ref{fig:densitysex} shows the distributions of log hourly wages by sector and sex.  The public sector pay premium for women is much higher than that enjoyed by men, at 39.2 per cent and 18.5 per cent, respectively.

\begin{figure}[tb]
    \centering
    \includegraphics[scale=.5]{DensitySex.pdf}
    \caption{Kernel density estimates of the distribution of public and private sector wages split by sex\label{fig:densitysex}}
\end{figure}

\section{Regression results}
The set of estimates shown in Table \ref{tab:baseline} are from random intercept regressions with the natural log of hourly wages as dependent variable and only an indicator of public sector status, sex, and the set of wave fixed effects included as regressors, although these last effects are not shown in the table.  The difference between the average earnings of men and women in the private and public sectors in each socio-economic group is shown graphically in figure \ref{fig:base}. The percentage pay premiums are shown in table \ref{tab:baseprem}.  Perhaps the most striking feature of these results is the discrepancy between the pay gaps experienced by men and women.  In all cases, the premium is higher for women than men, although the difference is not statistically significant for higher supervisory and lower technical occupations.


\begin{table}[tb]
\caption{Random intercept regression estimates (standard errors in parentheses).\label{tab:baseline}}
\begin{center}
\begin{tabular}{lcccccc}
\toprule
            &Routine            &Semi-routine &Lower &Intermediate    &Higher \\
            &       &   & technical & & supervisory \\
\midrule
Constant	        &1.42 	            &1.42 	   &1.44	  	&1.47 	   &1.52\\
					&(.024)	            &(.021)	   &(.029)	    &(.022)	   &(.059)\\
Public sector       &.311		        &.241	   &.289		&.302	   &.331\\
					&(.017)	            &(.015)	   &(.020)	    &(.016)	   &(.037)\\
Male			    &.294				&.299	   &.313        &.311	   &.348\\
					&(.015)				&(.013)	   &(.017)	    &(.014)	   &(.031)\\
Public sector       & -.122				&-.079	   &-.055		&-.116	   &-.041\\
\quad $\times$ male &(.028)		        &(.024)	   &(.032)		&(.026)	   &(.057)\\
\midrule
N					&8367  	            &10963 	   &6734       &10059 	   &1852    \\
Deviance	        &11381.8	        &14255.2   &9641.5     &13379.5	   &2783.9\\
\\
\bottomrule
\end{tabular}

\begin{tabular}{lcccc}
\\
\toprule
			&Lower & Lower &Higher &Higher \\
            &managerial & professional & professional & managerial \\
\midrule
Constant	        &1.43	    &1.43	   &1.42	   	&1.53	      	\\
					&(.034)	    &(.025)	   &(.042)	    &(.048)	        \\
Public sector       &.362 	    &.377      &.320     	&.306 	        \\
                    &(.022)     &(.018)	   &(.028)	    &(.034)	        \\
Male				&.363		&.321	   &.375		&.342	        \\
					&(.019)		&(.015)	   &(.024)		&(.028)			\\
Public sector       &-.195		&-.172	   &-.218		&-.144			\\
\quad $\times$ male &(.035)     &(.029)	   &(.043)		&(.051)			\\
\midrule
N 		  	        &5002       &8633      &3797    	&2356          \\
Deviance	        &6912.4	    &12045.5   &5861.7     &3368.9         \\
\bottomrule
\end{tabular}
\end{center}
\end{table}

\begin{figure}[ht]
    \includegraphics[scale=.7,angle=90]{BasePremiumSEC.pdf}
    \caption{Estimated public sector premiums for men and women showing 95 per cent
    confidence intervals.\label{fig:base}}
\end{figure}

\begin{table}[ht]
\caption{Percentage public sector premiums for men and women in each socio-economic group \label{tab:baseprem}}
\begin{center}
\begin{tabular}{llr}
  \toprule
  Socio-economic group & Sex & Premium (\%) \\
  \midrule
  Higher management & Men & 17.61 \\
  Higher management & Women & 35.82 \\
  Higher professional & Men & 10.84 \\
  Higher professional & Women & 37.77 \\
  Lower professional & Men & 22.80 \\
  Lower professional & Women & 45.80 \\
  Lower managerial & Men & 18.13 \\
  Lower managerial & Women & 43.61 \\
  Higher supervisory & Men & 33.62 \\
  Higher supervisory & Women & 39.22 \\
  Intermediate occupations & Men & 20.46 \\
  Intermediate occupations & Women & 35.30 \\
  Lower technical & Men & 26.34 \\
  Lower technical & Women & 33.47 \\
  Semi-routine & Men & 17.58 \\
  Semi-routine & Women & 27.29 \\
  Routine & Men & 20.76 \\
  Routine & Women & 36.47 \\
\bottomrule
\end{tabular}
\end{center}
\end{table}

However, these estimates do not control for a number of important characteristics
that are likely to affect earnings and which also may well differ between the public and
private sectors. In table \ref{tab:full} we show estimates of random intercept models that control
for age, education, and seniority in addition to including sex and year fixed effects (again, not shown in the table). These results are displayed graphically in figure \ref{fig:full}. Including these controls results in a reduction in the estimated size of the public sector pay premiums (or penalty) compared to those shown in table \ref{tab:baseline}, as one would expect given what we know about the differences in the public and private sector workforces.  The largest pay premium---17.8 per cent---is enjoyed by female employees in lower managerial occupations closely followed by women lower professionals.
Double-digit public sector premiums are also experienced by women who work in higher management, higher professional, higher supervisory, intermediate, lower technical and routine occupations.  All of these estimates are statistically significant, as is the slightly smaller premium enjoyed by women working in semi-routine occupations.

\begin{table}[tb]
    \caption{Random intercept regression estimates (standard errors in parentheses).\label{tab:full}}
    \begin{tabularx}{\textwidth}{Yccccc}
    \toprule
                    &Routine    &Semi-routine &Lower &Intermediate    &Higher \\
                    &           &             &technical&           &supervisory \\
    \midrule
    Constant		&-.209     	&-.239		&-.252	&-.248	&-.173\\
    				&(.048)	    &(.042)	    &(.056)	&(.039)	&(.109)\\
    Public sector	&.128		&.089		&.099	&.127	&.117\\
    				&(.015)	    &(.013)	    &(.017)	&(.014)	&(.033)\\
    Male			&.258	    &.263	    &.261	&.263	&.280\\
    				&(.013)	    &(.011)	    &(.015)	&(.012)	&(.026)\\
    Public sector   &-.091	    &-.065	    &-.027	&-.073	&-.004\\
    \quad $\times	$ Male%
    				&(.024)	    &(.021)	    &(.028)	&(.022) &(.050)\\
    Age				&.077	    &.075    	&.078	&.079	&.077\\
    				&(.002)	    &(.002)	    &(.003)	&(.002)	&(.005)\\
    Age squared		&-.859	    &-.821	    &-.852	&-.860	&-.842\\
    \quad (in thousands)%	
    				&(.029)	    &(.026)	    &(.034)	&(.027) &(.067)\\
    Secondary 		&.202	    &.239     	&.245	&.233	&.214\\
    \quad education &(.016)     &(.014)	    &(.018)	&(.015)	&(.033)\\
    Tertiary		&.567	    &.586		&.635	&.604	&.608\\
    \quad education &(.019)     &(.016)	    &(.021)	&(.017)	&(.038)\\
    Seniority		&.005	    &.005		&.004	&.003	&.002\\
    				&(.001)	    &(.001)	    &(.001)	&(.001)	&(.002)\\
    \midrule
    Std. Dev. Constant&	.335    &.344       &.368	&.359	&.354\\
    Std. Dev. Residual&	.292    &.273       &.275	&.266	&.285\\
    N			    &8240       &10789      &6624	&9883	&1826\\
    Individuals		&5888       &7237       &4986	&6773	&1653\\
    Deviance		&8956.5     &10979.7    &7502.2 &10263.2&2192.2\\
    \bottomrule
    \end{tabularx}
\end{table}

\begin{table}[t]
    \begin{tabularx}{\textwidth}{Ycccc}
    \toprule
			&Lower    & Lower &Higher &Higher \\
            & managerial & professional & professional & managerial \\
    \midrule
    Constant			&-.301	&-.355	&-.306	&-.026	   \\
    					&(.065)	&(.050)	&(.081)	&(.094)	    \\
    Public sector		&.164	&.163	&.132	& .136	\\
    					&(.019)	&(.016)	&(.025)	&(.030)	    \\
    Male				&.298	&.268	&.311	&.309	\\
    					&(.016)	&(.013)	&(.021)	&(.024)	\\
    Public sector		&-.141	&-.110	&-.161	&-.133	\\
    \quad $\times$ Male%
    					&(.030) &(.025)	&(.038)	&(.044)	\\
    Age					&.078	&.082	&.079	&.065	\\
    					&(.003)	&(.002)	&(.004)	&(.004)	\\
    Age squared			&-.861	&-.917	&-.888	&-.683	\\
    \quad (in thousands)%
    					&(.039) &(.030)	&(.048)	&(.055)	\\
    Secondary 			&.271	&.240	&.266	&.244	\\
    \quad education	    &(.020)	&(.017)	&(.025)	&(.030)	\\
    Tertiary			&.632	&.605	&.643	&.643	\\
    \quad education	    &(.023)	&(.019)	&(.029)	&(.033)	 \\
    Seniority			&.004	&.005	&.005	&.003	\\
    					&(.001)	&(.001)	&(.001)	&(.002)	\\
    \midrule
    Std. Dev. Constant  &.353   &.355	&.394	&.390	\\
    Std. Dev. Residual  &.264   &.285	&.287	&.225	\\
    N					&4936   &8508	&3752	&2306	\\
    Individuals			&3967   &5996	&3076	&1995	\\
    Deviance			&5340.4 &9445.4 &4825.6 &2568.5   \\
    \bottomrule
    \end{tabularx}
\end{table}

Undoubtedly the most striking finding, however, is that for men the public sector pay premium almost disappears.  In five occupations, the difference in pay between public and private sector employees is not statistically significant.  In the higher supervisory, intermediate, lower professional and lower technical occupations there is a statistically significant difference, although in all but one of these the estimated premium is less than ten per cent.  On the other hand, even controlling for these individual differences, women in the public sector continue to earn large and statistically significant pay premiums in all occupational groups.

The implication of this is that the public sector pay premium is not just higher for women than men, as found in some previous studies, but is in fact largely confined to women.  This naturally raises the question of why there should be such a striking gender difference.

\begin{table}[ht]
    \caption{Percentage public sector premiums for men and women in each socio-economic group. \label{tab:fullprem}}
    \begin{center}
    \begin{tabular}{llr}
\toprule
     Socio-economic group & Sex & Premium (\%) \\
\midrule
  Higher management & Men & 0.31 \\
  Higher management & Women & 14.61 \\
  Higher professional & Men & -2.82 \\
  Higher professional & Women & 14.12 \\
  Lower professional & Men & 5.42 \\
  Lower professional & Women & 17.68 \\
  Lower managerial & Men & 2.29 \\
  Lower managerial & Women & 17.80 \\
  Higher supervisory & Men & 12.04 \\
  Higher supervisory & Women & 12.45 \\
  Intermediate occupations & Men & 5.57 \\
  Intermediate occupations & Women & 13.55 \\
  Lower technical & Men & 7.51 \\
  Lower technical & Women & 10.43 \\
  Semi-routine & Men & 2.41 \\
  Semi-routine & Women & 9.29 \\
  Routine & Men & 3.75 \\
  Routine & Women & 13.63 \\
\bottomrule
    \end{tabular}
    \end{center}
\end{table}

\begin{figure}
    \includegraphics[scale=.7,angle=90]{FullPremiumSEC.pdf}
    \caption{Estimated pay premiums controlling for age, qualifications and seniority,
    showing 95 per cent confidence intervals.\label{fig:full}}
\end{figure}

Given the eighteen year period over which the data analysed in this paper were collected, it is important to check for time variation in the size of the public sector premiums and penalties.  To achieve this, interactions between wave indicators and the public sector dummy variable were created.  Full results of these regressions are available from the author.  Tests of a significant improvement in model fit as a result of including these interactions are shown in table \ref{tab:time}.  These show that there is evidence of statistically significant time variation in the size of the public sector premium in six of the SEGs.  However, as figure \ref{fig:time} shows, these differences do not substantially change the patterns discussed above.

\begin{table}[ht]
    \caption{Tests of time invariance. \label{tab:time}}
    \begin{center}
    \begin{tabular}{lrr}
    \toprule
    Socio-economic group & Chi-square & p-value \\
    \midrule
    Personal service & 31.5 & .017 \\
    Unskilled manual & 13.3 & .71 \\
    Junior non-manual & 28.7 & .037 \\
    Semi-skilled manual & 22.5 & .167 \\
    Skilled manual & 29.8 & .028 \\
    Foreman manual & 10.5 & .88 \\
    Managers, small & 21.3 & .21 \\
    Int. non-manual & 43.6 & $ < .01 $\\
    Managers, large&30.3 & .024 \\
    Professionals & 52.2 & $< .01$ \\
    \bottomrule
    \end{tabular}
    \end{center}
\end{table}

\begin{figure}[ht]
    \includegraphics[scale=.7,angle=90]{TimeVariationSEC.pdf}
    \caption{Time variation in the public sector premium. \label{fig:time}}
\end{figure}

\section{Discussion}
These results demonstrate that controlling for key indicators of individual human capital, when we compare \emph{male} public and private sector employees in similar occupations most of the difference in earnings that has been frequently reported in previous research disappears.  However, the same is not true of women, and this raises the question, why do women fare so much worse in the private sector than their male counterparts?

Some of the difference may be due to the gendered nature of many occupations.  Since occupations are not evenly distributed between the private and public sectors, if female-dominated occupations are more likely to be found in the private sector, then we might observe this pattern, given that wage levels in female-dominated occupations have been shown to be lower.  To check this we calculated the percentage of the employees in an occupation in our sample that were women.  The mean percentage in the public sector was 63 per cent, while in the private sector is was 45 per cent. Of all the women employees in the sample, 66 per cent worked in the public sector.  So in fact there are more jobs in female-dominated occupations in the public sector than in the private sector.  In a sense this is to be expected; if women can earn more in the public sector while men enjoy no such premium, we would expect a much stronger preference for working in the public sector among women than men.  However, this makes it less likely that the observed public sector premium for women is due to differences in the occupational characteristics of the two sectors.

\begin{db}
Other ideas?
\end{db}

\clearpage
\bibliography{Employment}
\end{document}
